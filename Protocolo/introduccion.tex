\vspace*{\fill}


\section{Introducción}



La tomografía computarizada (TC) ha revolucionado el campo de la medicina, proporcionando imágenes tridimensionales detalladas del interior del cuerpo humano, lo que permite diagnósticos más precisos y tratamientos más eficaces. Esta técnica se basa en la generación y detección de rayos X que atraviesan el cuerpo, formando imágenes secciónales que pueden ser reconstruidas para ofrecer una visión completa y no invasiva de estructuras internas. La TC ha demostrado ser indispensable en diversas especialidades médicas, como la neurología, cardiología, y oncología, entre otras, donde la visualización detallada de tejidos y órganos es crucial para la toma de decisiones clínicas. \\

Tradicionalmente, los rayos X utilizados en la TC se generan a través del proceso de Bremsstrahlung, un fenómeno físico en el que electrones acelerados son frenados al pasar cerca de núcleos atómicos, resultando en la emisión de radiación en el rango de los rayos X. Este proceso ha sido la base de la tecnología de TC durante décadas, proporcionando una fuente confiable y efectiva de rayos X. Sin embargo, este método también tiene limitaciones inherentes, como la dispersión no deseada de rayos X, que puede reducir la calidad de la imagen, y la dosis relativamente alta de radiación necesaria para obtener imágenes de alta resolución. \\

Con el avance de las tecnologías y la creciente necesidad de mejorar la calidad de las imágenes al mismo tiempo que se reduce la exposición a la radiación, se ha iniciado la exploración de métodos alternativos para la generación de rayos X. Uno de estos métodos es la retrodispersión de Thomson, un proceso en el cual fotones interactúan con electrones libres o ligados débilmente, siendo dispersados hacia atrás con una energía que depende del ángulo de dispersión. Este enfoque podría ofrecer varias ventajas sobre el Bremsstrahlung, como una mayor precisión en la formación de imágenes y la posibilidad de reducir la dosis de radiación administrada al paciente, un aspecto crucial en la radiología moderna. \\

En este trabajo se evaluará la viabilidad de la retrodispersión de Thomson como una alternativa al método convencional de Bremsstrahlung en la tomografía computarizada. Para ello, se realizarán simulaciones comparativas utilizando el toolkit \textit{Geant4}, una herramienta ampliamente utilizada en física de partículas y radiación, que permitirá modelar con precisión los procesos involucrados en ambos métodos. Estas simulaciones no solo proporcionarán una comparación detallada de las características de los rayos X generados por ambos métodos, sino que también ayudarán a determinar el impacto potencial en la calidad de la imagen y la dosis de radiación en el contexto clínico. Con este estudio, se espera contribuir al desarrollo de nuevas tecnologías en la TC que mejoren tanto la seguridad como la efectividad de esta herramienta diagnóstica crucial.


\vspace*{\fill}