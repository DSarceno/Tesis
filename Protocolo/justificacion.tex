\vspace*{\fill}


\section{Justificación}


%La mejora continua de las técnicas de diagnóstico por imágenes es de suma importancia en el campo de la medicina, donde la precisión y la seguridad del paciente son primordiales. La tomografía computarizada, aunque es una herramienta invaluable, conlleva riesgos asociados a la exposición a radiación. Por ello, la exploración de métodos alternativos de generación de rayos X, como la retrodispersión de Thomson, es crucial para, potencialmente, reducir la dosis de radiación y mejorar la calidad de las imágenes. \\

%\textit{Geant4}, un software ampliamente reconocido en física de partículas y física médica, ofrece la capacidad de modelar y comparar diferentes métodos de generación de viabilidad técnica de la retrodispersión de Thomson, sino que también contribuirá al desarrollo de futuras aplicaciones en el ámbito médico, pudiendo influir en la práctica clínica y en la salud pública.

Realizar este proyecto es crucial debido a la necesidad constante de mejorar las técnicas de diagnóstico por imágenes en medicina. La tomografía computarizada, aunque fundamental, conlleva riesgos asociados a la exposición a radiación, lo que subraya la importancia de explorar métodos alternativos de generación de rayos X. La retrodispersión de Thomson emerge como una opción prometedora que podría no solo reducir la dosis de radiación, sino también mejorar la precisión de las imágenes, beneficiando así la seguridad y el bienestar de los pacientes. \\

Implementar este estudio en \textit{Geant4}, un software reconocido por su precisión en simulaciones de física de partículas y aplicaciones médicas, ofrece una plataforma robusta para modelar y evaluar la viabilidad técnica de la retrodispersión de Thomson. Esta herramienta no solo permitirá comparar este método con los convencionales, sino que también posibilitará el desarrollo de nuevas aplicaciones en el campo médico. Esto tiene el potencial de influir significativamente en la práctica clínica, mejorando los estándares de diagnóstico y tratamiento, así como en la salud pública al reducir la exposición a la radiación de los pacientes y profesionales de la salud. \\

Realizar este proyecto no solo se justifica por su impacto potencial en el campo médico, sino también por mi capacidad para llevarlo a cabo de manera efectiva. Con un enfoque en física y experiencia en simulaciones computacionales, estoy bien posicionado para utilizar herramientas como \textit{Geant4} y explorar a fondo la retrodispersión de Thomson. Mi compromiso con la investigación innovadora y la mejora continua en salud me motiva a llevar a cabo este proyecto, aplicando mis habilidades para contribuir al avance de técnicas de diagnóstico más seguras y precisas para beneficio de la comunidad médica y de los pacientes.



\vspace*{\fill}