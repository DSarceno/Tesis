\vspace*{\fill}


\section{Objetivos}

\subsection{General}
Evaluar la viabilidad y efectividad de la Dispersión de Thomson como método para la generación de rayos X en tomografía computarizada, comparandolo con el método tradicional de Bremsstrahlung, mediante simulaciones de \textit{Geant4}.

\subsection{Específicos}
\begin{enumerate}
    \item Desarrollar un modelo teórico detallado del proceso de Dispersión de Thomson y su implementación en simulaciones de \textit{Geant4}.
    \item Realizar simulaciones en \textit{Geant4} de tomografía computarizada utilizando Bremsstrahlung y Dispersión de Thomson para generar rayos X.
    \item Comparar los resultados de las simulaciones de ambos métodos, evaluando eficiencia, dosis de radiación y precisión.
    \item Implementar y comparar la reconstrucción de imágenes tomográficas obtenidas con cada método, evaluando la calidad y resolución de las imágenes.
\end{enumerate}


\vspace*{\fill}