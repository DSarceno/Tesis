\section{Índice Preliminar}


\begin{enumerate}
    \item Capitulo 1: Teoría sobre la Dispersión de Thomson.
    \begin{enumerate}
        \item Potencial de Liènard-Wiechert.
        \begin{enumerate}
            \item Solución a la ecuación de movimiento del electrón.
            \item Espectro de la Dispersión de Thomson.
            \item Fórmula de Larmor.
        \end{enumerate}
        \item Introducción a \emph{Wakefield Acceleration} y a dinámica relativista.
        \item Ejemplos y Aplicaciones extras de la Dispersión de Thomson.
    \end{enumerate}
    \item Capitulo 2: ¿Qué es la Tomografía Computarizada?
    \begin{enumerate}
        \item Historia e Importancia de la Tomografía Computarizada.
        \begin{enumerate}
            \item Primeros pasos y evolución de la TC.
            \item Aplicaciones Médicas.
            \item Usos en Investigación.
        \end{enumerate}
        \item Principios Básicos
        \begin{enumerate}
            \item Interacción de Rayos X con la materia.
            \begin{enumerate}
                \item Atenuación y Absorción de Rayos X.
                \item Coeficiente de Atenuación Lineal.
            \end{enumerate}
        \end{enumerate}
        \item Componentes y Tecnología.
        \begin{enumerate}
            \item Fuente de Rayos X.
            \begin{enumerate}
                \item Generación de Rayos X: Bremsstrahlung.
                %y emisión característica
                \item Espectro de Rayos X y filtración.
            \end{enumerate}
            \item Algoritmo de Reconstrucción
        \end{enumerate}
    \end{enumerate}
    \item Capitulo 3: Simulaciones en \textit{Geant4} de Tomografía Computarizada.
    \begin{enumerate}
        \item ¿Qué es \textit{Geant4}?
        \begin{enumerate}
            \item Descripción General de \textit{Geant4}.
            \item Aplicaciones en Física Médica.
            \item Precisión en Simulaciones de Interacciones Radiación-Materia.
        \end{enumerate}
        \item Descripción de los Métodos Simulados.
        \begin{enumerate}
            \item Método Tradicional de Generación de Rayos X: Bremsstrahlung.
            \item Generación de Rayos X mediante Dispersión de Thomson.
            \begin{enumerate}
                \item Configuración y Modelado en \textit{Geant4}.
            \end{enumerate}
        \end{enumerate}
        \item Metodología para la comparación de resultados.
        \begin{enumerate}
            \item Diseño de la Simulación en \textit{Geant4}.
            \item Criterios de Evaluación.
        \end{enumerate}
        \item Cuantificación de Parámetros Importantes.
        \begin{enumerate}
            \item Calidad de Imagen.
            \item Dosimetría.
        \end{enumerate}
    \end{enumerate}
    \item Capitulo 4: Reconstrucción de Imágenes.
    \begin{enumerate}
        \item Importancia y Conceptos Básicos.
        \begin{enumerate}
            \item Impacto en la Precisión Diagnóstica.
            \item Proyección y Retroproyección.
            \item Concepto de Imagen Tomográfica.
        \end{enumerate}
        \item Reconstrucción de Imágenes
        %\item Algoritmos de Reconstrucción.
        %\begin{enumerate}
        %    \item Retroproyección Filtrada (FBP).
        %    \item Transformada de Radón.
        %    \item Reconstrucción Iterativa.
        %\end{enumerate}
    \end{enumerate}
\end{enumerate}
