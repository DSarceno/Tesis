\vspace*{\fill}

\section{Metodología}


\begin{description}
    \item[Revisión Bibliográfica: ] Se realizará una revisión de la literatura existente sobre la generación de rayos X mediante bremsstrahlung y retrodispersión de Thomson, así como un poco sobre \textit{wakefield acceleration}.
    \item[Modelado Teórico: ] Se explorará la teoría necesaria para describir el efecto de retrodispersión de Thomson, detallando los principios físicos que lo sustentan. Esta exploración teórica permitirá entender cómo se genera y cómo se puede aplicar en la simulación computacional para mejorar la generación de rayos X en tomografía.
    \item[Simulaciones en \textit{Geant4}: ] Utilizando \textit{Geant4}, se llevarán a cabo simulaciones de tomografía computarizada con rayos X generados tanto por bremsstrahlung como por retrodispersión de Thomson. Se analizarán parámetros clave como la resolución de imágenes, la dosis de radiación y la eficiencia de ambos métodos.
    \item[Análisis Comparativo: ] Los resultados de las simulaciones serán comparados para determinar ventajas y desventajas de cada método en base a los parámetros mencionados anteriormente.
\end{description}


\vspace*{\fill}