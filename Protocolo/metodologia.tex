\vspace*{\fill}

\section{Metodología}


\begin{description}
    \item[Revisión Bibliográfica: ] se realizará una revisión de la literatura existente sobre el fenómeno a estudiar por medio de artículos específicos como el de S. Corde et al. \cite{corde2013femtosecond} o el raelizado por Tajima, Toshiki y Dawson \cite{tajima1979laser}; así como de libros especializados como el de John D. Jackson \cite{jackson2021classical} o el de Gabor T. Herman \cite{herman2009fundamentals}.
    \item[Modelado Teórico: ] se explorará la teoría necesaria para describir el efecto de Dispersión de Thomson, detallando los principios físicos que lo sustentan. Esta exploración teórica permitirá entender cómo se genera y cómo se puede aplicar en la simulación computacional para mejorar la generación de rayos X en tomografía.
    \item[Simulaciones en \textit{Geant4}: ] utilizando \textit{Geant4}, se llevarán a cabo simulaciones de tomografía computarizada con rayos X generados tanto por Bremsstrahlung como por Dispersión de Thomson. Se analizarán parámetros clave como la resolución de imágenes, la dosis de radiación y la eficiencia de ambos métodos.
    \item[Análisis Comparativo: ] los resultados de las simulaciones serán comparados para determinar ventajas y desventajas de cada método en base a los parámetros mencionados anteriormente.
\end{description}


\vspace*{\fill}